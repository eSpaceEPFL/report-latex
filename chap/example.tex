\section{Example document}

    This section serves as example to demonstrate appearance of the template. And give information on the way it works.

    \subsection{File and folder hierarchy}

        This template has a preset folder and file hierarchy to have a clear structure. If you know what you're doing, you can play around with it but it works well as is.

        \subsubsection{Chapters and section and appendices}

            It is recommended to place all your sections as separate \texttt{.tex} files and store them in the chap folder. It is the referenced in the main doucment with \verb'\section{General document guidelines}

    This section serves as example to demonstrate appearance of the template. And give information on the way it works.

    \subsection{File and folder hierarchy}

        This template has a preset folder and file hierarchy to have a clear structure. If you know what you're doing, you can play around with it but it works well as is.

        \subsection{Chapters and section and appendices}

            It is recommended to place all your sections as separate \texttt{.tex} files and store them in the chap folder. It is the referenced in the main doucment with \verb'\section{General document guidelines}

    This section serves as example to demonstrate appearance of the template. And give information on the way it works.

    \subsection{File and folder hierarchy}

        This template has a preset folder and file hierarchy to have a clear structure. If you know what you're doing, you can play around with it but it works well as is.

        \subsection{Chapters and section and appendices}

            It is recommended to place all your sections as separate \texttt{.tex} files and store them in the chap folder. It is the referenced in the main doucment with \verb'\section{General document guidelines}

    This section serves as example to demonstrate appearance of the template. And give information on the way it works.

    \subsection{File and folder hierarchy}

        This template has a preset folder and file hierarchy to have a clear structure. If you know what you're doing, you can play around with it but it works well as is.

        \subsection{Chapters and section and appendices}

            It is recommended to place all your sections as separate \texttt{.tex} files and store them in the chap folder. It is the referenced in the main doucment with \verb'\input{chap/example}'.

            Appendices work on the same princile and can be stores in the appendices folder.

        \subsection{Figures}

            Figures are stored in the fig folder and \LaTeX will automatically look for the image file there so you should reference it from this folder. A special function was implemented to allow the insertion of a single figure with a single line of code and have the filename, the label, the legend and the width defined.

            {\scriptsize\verb'\figi{cleanspace-one}{fig:cleanspace-one}{Image of CleanSpace One}{0.4\textwidth}'}

            \figi{cleanspace-one}{fig:cleanspace-one}{Image of CleanSpace One}{0.4\textwidth}


    \subsection{Bibliography}

        The system has been simplified and uses Bibtex as a compiler. All documents are in \texttt{refdoc.bib}.
        
        
           
\section{Mission objectives} 

\subsection{Problem setting}

\subsection{State-of-the-art}

\subsection{Key driving requirements} 


\section{Science and Instruments}
\subsection{Definition of science objectives}
\subsection{Measurement objectives}
\subsection{Definition of science instruments}
\subsection{Definition of pointing requirements}
\subsection{Data volume, data rate}
\subsection{Strategy for data collection}


\section{Mission Definition} 
\subsection{Mission type}

\subsection{Launch date}

\subsection{Mission duration}

\subsection{Mission objective}

\subsection{Mission class}

\subsection{Redundancy}

\subsection{Other mission constraints}

\subsection{Heritage}

\section{Mission Design} 

\subsection{Requirements}

\subsection{Orbital elements of selected final orbit}

\subsection{Eclipses, orbital period, coverage}

\subsection{Maneuvers, DV budget}

\subsection{Planetary insertion}

\subsection{Launch vehicle selected, fairing}

\subsection{Launch site}

\subsection{LV capability}

\subsection{Mission operations scenario at mission level}
Also known as Concept of Operations (CONOPS)

\section{Space environment}
\subsection{Total radiation dose}
\subsection{Planetary environment }

\section{Telecom}
\subsection{UL, DL frequency and data rate}
\subsection{Modulation, protocol}
\subsection{Ground stations selected}
\subsection{Link budget}
\subsection{Telecom hardware + Antenna design / power amplifiers}

\section{Electrical power}
\subsection{Power budget}
\subsection{Sizing solar arrays, batteries}
\subsection{EPS architecture}

\section{Command and Data Handling} 
\subsection{Requirements}
\subsection{Data storage}
\subsection{Data bus}
\subsection{Data budget}
\subsection{Protocols, encoding}

\section{Attitude Determination and Control System} 
\subsection{Requirements}
\subsection{Stabilization method}
\subsection{Pointing control, stability}
\subsection{Pointing DV budgets}

\section{Propulsion}
\subsection{Requirements}
\subsection{Choice of thrusters}
\subsection{Propellant budget}

\section{Structure and Configuration} 
\subsection{Requirements}
\subsection{Main structure design, launch adapter}
\subsection{Main hardware boxes and elements configuration}
\subsection{Mass properties}

\section{Thermal} 
\subsection{Requirements}
\subsection{Thermal balance analysis}
\subsection{Thermal design}
\subsection{Thermal requirements}

\section{System Engineering} 
\subsection{High level requirements}
\subsection{Mass budget}
\subsection{Power budget}
\subsection{Data budget}
\subsection{System Product Breakdown Structure}
\subsection{Funciomnal diagram} 
\subsection{Redundancy scheme}
'.

            Appendices work on the same princile and can be stores in the appendices folder.

        \subsection{Figures}

            Figures are stored in the fig folder and \LaTeX will automatically look for the image file there so you should reference it from this folder. A special function was implemented to allow the insertion of a single figure with a single line of code and have the filename, the label, the legend and the width defined.

            {\scriptsize\verb'\figi{cleanspace-one}{fig:cleanspace-one}{Image of CleanSpace One}{0.4\textwidth}'}

            \figi{cleanspace-one}{fig:cleanspace-one}{Image of CleanSpace One}{0.4\textwidth}


    \subsection{Bibliography}

        The system has been simplified and uses Bibtex as a compiler. All documents are in \texttt{refdoc.bib}.
        
        
           
\section{Mission objectives} 

\subsection{Problem setting}

\subsection{State-of-the-art}

\subsection{Key driving requirements} 


\section{Science and Instruments}
\subsection{Definition of science objectives}
\subsection{Measurement objectives}
\subsection{Definition of science instruments}
\subsection{Definition of pointing requirements}
\subsection{Data volume, data rate}
\subsection{Strategy for data collection}


\section{Mission Definition} 
\subsection{Mission type}

\subsection{Launch date}

\subsection{Mission duration}

\subsection{Mission objective}

\subsection{Mission class}

\subsection{Redundancy}

\subsection{Other mission constraints}

\subsection{Heritage}

\section{Mission Design} 

\subsection{Requirements}

\subsection{Orbital elements of selected final orbit}

\subsection{Eclipses, orbital period, coverage}

\subsection{Maneuvers, DV budget}

\subsection{Planetary insertion}

\subsection{Launch vehicle selected, fairing}

\subsection{Launch site}

\subsection{LV capability}

\subsection{Mission operations scenario at mission level}
Also known as Concept of Operations (CONOPS)

\section{Space environment}
\subsection{Total radiation dose}
\subsection{Planetary environment }

\section{Telecom}
\subsection{UL, DL frequency and data rate}
\subsection{Modulation, protocol}
\subsection{Ground stations selected}
\subsection{Link budget}
\subsection{Telecom hardware + Antenna design / power amplifiers}

\section{Electrical power}
\subsection{Power budget}
\subsection{Sizing solar arrays, batteries}
\subsection{EPS architecture}

\section{Command and Data Handling} 
\subsection{Requirements}
\subsection{Data storage}
\subsection{Data bus}
\subsection{Data budget}
\subsection{Protocols, encoding}

\section{Attitude Determination and Control System} 
\subsection{Requirements}
\subsection{Stabilization method}
\subsection{Pointing control, stability}
\subsection{Pointing DV budgets}

\section{Propulsion}
\subsection{Requirements}
\subsection{Choice of thrusters}
\subsection{Propellant budget}

\section{Structure and Configuration} 
\subsection{Requirements}
\subsection{Main structure design, launch adapter}
\subsection{Main hardware boxes and elements configuration}
\subsection{Mass properties}

\section{Thermal} 
\subsection{Requirements}
\subsection{Thermal balance analysis}
\subsection{Thermal design}
\subsection{Thermal requirements}

\section{System Engineering} 
\subsection{High level requirements}
\subsection{Mass budget}
\subsection{Power budget}
\subsection{Data budget}
\subsection{System Product Breakdown Structure}
\subsection{Funciomnal diagram} 
\subsection{Redundancy scheme}
'.

            Appendices work on the same princile and can be stores in the appendices folder.

        \subsection{Figures}

            Figures are stored in the fig folder and \LaTeX will automatically look for the image file there so you should reference it from this folder. A special function was implemented to allow the insertion of a single figure with a single line of code and have the filename, the label, the legend and the width defined.

            {\scriptsize\verb'\figi{cleanspace-one}{fig:cleanspace-one}{Image of CleanSpace One}{0.4\textwidth}'}

            \figi{cleanspace-one}{fig:cleanspace-one}{Image of CleanSpace One}{0.4\textwidth}


    \subsection{Bibliography}

        The system has been simplified and uses Bibtex as a compiler. All documents are in \texttt{refdoc.bib}.
        
        
           
\section{Mission objectives} 

\subsection{Problem setting}

\subsection{State-of-the-art}

\subsection{Key driving requirements} 


\section{Science and Instruments}
\subsection{Definition of science objectives}
\subsection{Measurement objectives}
\subsection{Definition of science instruments}
\subsection{Definition of pointing requirements}
\subsection{Data volume, data rate}
\subsection{Strategy for data collection}


\section{Mission Definition} 
\subsection{Mission type}

\subsection{Launch date}

\subsection{Mission duration}

\subsection{Mission objective}

\subsection{Mission class}

\subsection{Redundancy}

\subsection{Other mission constraints}

\subsection{Heritage}

\section{Mission Design} 

\subsection{Requirements}

\subsection{Orbital elements of selected final orbit}

\subsection{Eclipses, orbital period, coverage}

\subsection{Maneuvers, DV budget}

\subsection{Planetary insertion}

\subsection{Launch vehicle selected, fairing}

\subsection{Launch site}

\subsection{LV capability}

\subsection{Mission operations scenario at mission level}
Also known as Concept of Operations (CONOPS)

\section{Space environment}
\subsection{Total radiation dose}
\subsection{Planetary environment }

\section{Telecom}
\subsection{UL, DL frequency and data rate}
\subsection{Modulation, protocol}
\subsection{Ground stations selected}
\subsection{Link budget}
\subsection{Telecom hardware + Antenna design / power amplifiers}

\section{Electrical power}
\subsection{Power budget}
\subsection{Sizing solar arrays, batteries}
\subsection{EPS architecture}

\section{Command and Data Handling} 
\subsection{Requirements}
\subsection{Data storage}
\subsection{Data bus}
\subsection{Data budget}
\subsection{Protocols, encoding}

\section{Attitude Determination and Control System} 
\subsection{Requirements}
\subsection{Stabilization method}
\subsection{Pointing control, stability}
\subsection{Pointing DV budgets}

\section{Propulsion}
\subsection{Requirements}
\subsection{Choice of thrusters}
\subsection{Propellant budget}

\section{Structure and Configuration} 
\subsection{Requirements}
\subsection{Main structure design, launch adapter}
\subsection{Main hardware boxes and elements configuration}
\subsection{Mass properties}

\section{Thermal} 
\subsection{Requirements}
\subsection{Thermal balance analysis}
\subsection{Thermal design}
\subsection{Thermal requirements}

\section{System Engineering} 
\subsection{High level requirements}
\subsection{Mass budget}
\subsection{Power budget}
\subsection{Data budget}
\subsection{System Product Breakdown Structure}
\subsection{Funciomnal diagram} 
\subsection{Redundancy scheme}
'.

            Appendices work on the same princile and can be stores in the appendices folder.

        \subsubsection{Figures}

            Figures are stored in the fig folder and \LaTeX will automatically look for the image file there so you should reference it from this folder. A special function was implemented to allow the insertion of a single figure with a single line of code and have the filename, the label, the legend and the width defined.

            {\scriptsize\verb'\figi{cleanspace-one}{fig:cleanspace-one}{Image of CleanSpace One}{0.4\textwidth}'}

            \figi{cleanspace-one}{fig:cleanspace-one}{Image of CleanSpace One}{0.4\textwidth}


    \subsection{Bibliography}

        The bibliography is a bit special and the compiler has to be adapted in order to work well. The idea is that there are two kinds of documents in your bibliography. \textit{Applicable documents} and \textit{Reference documents} and those would be stored inside two separate \texttt{.bib} files; \texttt{appdoc.bib} and \texttt{refdoc.bib} respectively. They can be cited in the document with the \verb'\cite{label}' command and gives \cite{Abedon1994} or \cite{AbedonHymanThomas2003}.

        \subsubsection{Modifiying the bibliography compiler}

            This template uses \texttt{Biber} and the compiler (at least for TexLive) has to be modified. A new compiler has to be created called in \texttt{Edit -> Preferences} then in tab called \texttt{Typesetting} under \texttt{Processing tools} click on the + sign and add a new one.

            The parameters should be :

            \begin{description}
                \item[Name] Biber
                \item[Program] biber
                \item[Arguments] \verb'$basename'
            \end{description}

            Or as summarized on figure \ref{fig:compiler-config}.

            \figi{compiler-config}{fig:compiler-config}{How the compiler should be configured}{0.5\textwidth}



