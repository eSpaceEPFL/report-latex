\section{Kill Switch design}

\subsection{Mechanical constraints}

    The mechanical constraints are the most problematic ones for the design of the Kill Switch. The space that is usually used for that function on CubeSat is already occupied by the cylinder and the screws holding the slices together. This forces the KS mechanism to be really small and contained.

    The cavities inside the cylinder define the total travel distance that is allowed for the KS and also the width of the system. It can also be used to guide the mechanism. Small alteration can be made to this part in order to accommodate the KS.

    The height of the PCB from the bottom of the satellite on which the SLCs are mounted is also an important factor as well as the SLC itself. The stroke distance of the SLC is also to be taken into account.

    Finally the M4 titanium screw holding the slices together has a hole allowing the KS mechanism to reach the bottom of the satellite. The diameter of the hole can also be adapted for optimal design.

\subsection{Considerations}



